%*********************************************************************************%
\chapter*{Societal summary}
\addcontentsline{toc}{chapter}{Societal summary}
\markboth{Societal summary}{Societal summary}

In the core of stars light atomic nuclei combine to form larger nuclei, a process called nuclear fusion, releasing large amounts of energy. If one were able to harness all the energy, 1 milligramme of fusion fuel (e.g. hydrogen) would release roughly the same amount of energy as 10 kilogrammes of gasoline. Besides this high efficiency of the fusion fuel, it furthermore does not release greenhouse gases as it burns, making it an attractive energy source. However, the conditions required for nuclear fusion to occur are quite extreme, with the fuel temperature close to 150 million degrees centigrade. At such temperatures, the fuel is a fully ionised plasma in which electrons are detached from their atomic nuclei. This plasma is not unlike a ``charged fluid '', and it responds to electromagnetic fields. \par  
One needs a way of confining the plasma, so that the hot plasma does not come into contact with a cool material wall. Since the plasma responds to magnetic fields, one can use the magnetic force to confine the plasma, and this method of attaining nuclear fusion is called magnetic confinement fusion. Note that in order to generate the magnetic field one requires large superconducting coils which need to be cooled to a few degrees Kelvin. We thus have a plasma of some 150 million degrees centigrade, confined with superconducting coils which are close to zero Kelvin, and these are within metres of each other. This large gradient can be sustained only if the plasma sufficiently insulating. However, these large gradients also give rise to turbulence. \par
This turbulence is similar to pouring milk into a cup of tea; the differences in velocity, density, and temperature between milk and tea give rise to all kinds of swirls and whirls, ultimately mixing the milk and the tea into a homogeneous whole. So, too, does the turbulence in the fusion reactor mix the hot and cold regions, which makes it more difficult to sustain the desired 150 million degrees centigrade. Understandably, much research has been devoted to understanding how turbulence arises in these reactors, so that we may control and suppress it. \par 
In this thesis, we tackle this problem by taking a step back from all the complex details of the turbulence and instead looking at budgets of energy. That is, we ask ourselves the question how much energy can at most reside in the all swirls of the turbulence, without actually considering what the whirly patterns may look like. This enables one to find easy-to-calculate expressions, and we subsequently compare this calculation of the energy-budget with actual turbulence as computed by a supercomputer. We find that there is correspondence between the two, showcasing that there is some structure in the turbulent chaos. This makes it possible to, computationally speaking, cheaply estimate the levels of turbulence in a fusion reactor, which could aid in designing better reactors in the long run. 
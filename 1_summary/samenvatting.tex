%*********************************************************************************%
\chapter*{Samenvatting}
\addcontentsline{toc}{chapter}{Samenvatting}
\markboth{Samenvatting}{Samenvatting}

Kernfusie, het proces waarbij lichte atoomkernen samenkomen om zwaardere kernen te vormen, geeft grote hoeveelheden energie vrij, waarbij de specifieke energie (de energie per eenheid massa brandstof) ongeveer een factor $\sim 10^7$ groter is dan die van benzine. Bovendien stoot dit proces (dat veel van de sterren in ons heelal aandrijft) geen broeikasgassen uit, en deze eigenschappen maken het tot een aantrekkelijke energiebron. Magnetische opsluitingsfusie probeert dit doel te bereiken door de hete brandstof, een geïoniseerd plasma bij ongeveer 150 miljoen graden Celsius, te omringen met een sterke magnetische veldsterkte van enkele Tesla. Belangrijk is dat deze brandstof goed geïsoleerd moet zijn om de hoge vereiste temperatuur te behouden. Echter, het plasma vertoont turbulentie, wat de isolatie verslechtert en het moeilijker maakt om de gewenste eigenschappen te bereiken. \par
In deze scriptie onderzoeken we de waarde van een thermodynamische grootheid, genaamd beschikbare energie (\AE{}), als een maat voor turbulentie in fusieplasma's. Belangrijk is dat een dieper inzicht in dergelijke turbulentie kan helpen bij het verbeteren van de prestaties van fusiereactoren, hopelijk dichter bij de realiteit van magnetische opsluitingsfusie brengend. Beschikbare energie dankt zijn naam aan het feit dat het een bovengrens is voor de hoeveelheid thermische energie die uit het plasma kan worden onttrokken, onderhevig aan een aantal beperkingen, d.w.z. het is de hoeveelheid energie die {\it beschikbaar} is om te worden vrijgegeven. Hoewel deze grootheid bijna een eeuw geleden voor het eerst verscheen in de wiskunde, heeft het als maat voor turbulentie relatief weinig toepassing gevonden. In deze scriptie doen we een eerste poging tot een dergelijk onderzoek door ons te specialiseren in zogenaamde gevangen elektronen, die gevangen zitten in een gebied met een zwakkere magnetische veldsterkte. De gebieden waarin deze deeltjes gevangen zitten, worden bepaald door het magnetische veld $\boldsymbol{B}$, en de functionele vorm van dit veld wordt vaak de geometrie ervan genoemd. \par
Om de \AE{} te berekenen, ontwikkelen we eerst methoden en referentiewaarden om nauwkeurig hoeveelheden te berekenen die kenmerkend zijn voor dergelijke gevangen deeltjes, aangezien deze de berekening van de \AE{} zullen binnengaan. Met deze numerieke routines wordt de \AE{} van gevangen elektronen in algemene geometrieën analytisch afgeleid in de limiet van een fluxbuis, een smal domein rond een magnetische veldlijn die overal evenwijdig loopt aan het magnetische veld. Deze uitdrukking voor de \AE{} wordt vervolgens berekend voor een aantal geometrieën waarvoor niet-lineaire turbulentiesimulaties bestaan, en vergeleken met de resultaten van deze simulaties. Door de turbulente energiestroom (dat wil zeggen de hoeveelheid energie per eenheid tijd per eenheid oppervlakte die door het domein stroomt in de volledig turbulente toestand) te vergelijken met de \AE{}, ontdekken we dat de twee gecorreleerd zijn via een machtsverband, wat aangeeft dat \AE{} wijst op fundamentele eigenschappen van de turbulentie. Om de verschillende afhankelijkheden die \AE{} vertoont verder te onderzoeken, gaan we over op het specialiseren van tokamaks. Dit stelt ons in staat om gemakkelijk te onderzoeken hoe \AE{} verandert wanneer men de magnetische schuif, het drukgradient, de driehoekigheid en andere kenmerken die centraal staan in tokamaks, varieert. Dit reproduceert enkele bekende trends, zoals de zeer gunstige aard van negatieve magnetische schuif, en het stelt ons in staat om kwalitatieve uitspraken te doen over de voordelen van negatieve driehoekigheid. \par
Al met al vinden we dat er waarde zit in \AE{} als een maat voor turbulentie, waardoor er goedkope rekenmethoden ontstaan om turbulenieniveaus te schatten zonder de noodzaak van niet-lineaire simulaties. Het biedt ook inzicht in waarom sommige configuraties gunstiger zijn dan andere, omdat \AE{} gemakkelijk te interpreteren is. Kijkend naar de toekomst toe, zouden mogelijke generalisaties het nut van \AE{} voorbij gevangen elektronen kunnen uitbreiden, waardoor het een nuttig hulpmiddel wordt in meer algemene scenario's. Als zo'n generalisatie succesvol blijkt te zijn, kan het gemakkelijk geometrieën vinden waarin turbulentie sterk verminderd is, wat uiteindelijk fusie-energie dichter bij de realiteit kan brengen.


\vspace*{11pt}\noindent



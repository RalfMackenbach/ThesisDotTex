%*********************************************************************************%
\chapter*{Samenvatting}
\addcontentsline{toc}{chapter}{Samenvatting}
\markboth{Samenvatting}{Samenvatting}

Bij kernfusie, het proces waarbij lichte atoomkernen samensmelten tot zwaardere kernen, komt een grote hoeveelheid energie vrij, waarbij de specifieke energie (de energie per eenheid massa brandstof) ongeveer een factor $\sim 10^7$ groter is dan die van benzine. Bovendien stoot dit proces geen broeikasgassen uit, en deze eigenschappen maken het tot een aantrekkelijke energiebron. Een manier kernfusie mogelijk te maken is door de hete brandstof, een geïoniseerd plasma van ongeveer 150 miljoen graden Celsius, te omringen met een magnetische veld die enkele Tesla sterk is. Belangrijk is dat deze brandstof thermisch goed isolerend moet zijn om de hoge vereiste temperatuur te behouden. Echter, het plasma vertoont turbulentie, wat de isolatie verslechtert en het moeilijker maakt om de gewenste eigenschappen te bereiken. \par
In dit proefschrift onderzoeken we of een thermodynamische grootheid genaamd „beschikbare energie” de turbulentie in fusieplasma's kan voorspellen. Een dieper begrip van de turbulentie kan helpen bij het verbeteren van de prestaties van fusiereactoren, wat deze reactoren hopelijk dichter bij de realiteit brengt. Beschikbare energie dankt zijn naam aan het feit dat het een bovengrens is voor de hoeveelheid thermische energie die uit het plasma kan worden onttrokken, onderhevig aan een aantal beperkingen, d.w.z. het is de hoeveelheid energie die {\it beschikbaar} is om te worden vrijgegeven. Hoewel deze grootheid bijna een eeuw geleden voor het eerst verscheen in de wiskunde, heeft het als maatstaf voor turbulentie relatief weinig toepassing gevonden. In dit proefschrift doen we een eerste poging tot een dergelijk onderzoek door ons te specialiseren in zogenaamde gevangen elektronen, die gevangen zitten in een gebied met een zwakkere magnetische veldsterkte. De gebieden waarin deze deeltjes gevangen zitten, worden bepaald door het magnetische veld $\boldsymbol{B}$, en de functionele vorm van dit veld wordt vaak de „geometrie” genoemd. \par
Om de beschikbare energie te berekenen ontwikkelen we numerieke methoden en ijkpunten, zodat we nauwkeurig kenmerkende kwantiteiten voor gevangen deeltjes kunnen berekenen. Dit is nodig, omdat beschikbare energie afhangt van dergelijke kwantiteiten. Vervolgens leiden we de beschikbare energie van gevangen elektronen in algemene geometrieën analytisch af in de limiet van een fluxbuis; een smal domein rond een magnetische veldlijn die overal evenwijdig loopt aan het magnetische veld. Deze uitdrukking voor de beschikbare energie wordt vervolgens numeriek berekend voor een aantal geometrieën waarvoor niet-lineaire turbulentiesimulaties bestaan, en deze wordt vergeleken met de resultaten van de simulaties. De turbulente energiestroom (dat wil zeggen de hoeveelheid energie per eenheid tijd per eenheid oppervlakte die door het domein stroomt in de volledig turbulente toestand) blijkt gecorreleerd aan de beschikbare energie via een machtsverband, wat aangeeft dat beschikbare energie voorspellend vermogen heeft. Om de verschillende afhankelijkheden die beschikbare energie vertoont verder te onderzoeken gaan we over tot een specialisatie naar tokamaks. Dit stelt ons in staat om te onderzoeken hoe beschikbare energie verandert wanneer men de magnetische „shear”, de drukgradient, de driehoekigheid, en andere grootheden die kenmerkend zijn voor tokamaks, varieert. Dit reproduceert enkele bekende effecten, zoals de zeer gunstige aard van negatieve magnetische shear, en het stelt ons in staat om kwalitatieve uitspraken te doen over de voordelen van negatieve driehoekigheid. \par
Al met al vinden we dat beschikbare energie een voorspellend vermogen heeft voor turbulentie, waardoor er goedkope rekenmethoden ontstaan om turbulenieniveaus te schatten zonder noodzaak van niet-lineaire simulaties. Het biedt ook inzicht in waarom sommige configuraties gunstiger zijn dan andere, omdat beschikbare energie gemakkelijk te interpreteren is. Generalisaties zouden het domein van beschikbare energie voorbij gevangen elektronen kunnen uitbreiden, waardoor het een nuttig hulpmiddel wordt in meer algemene situaties. Als zo'n generalisatie succesvol blijkt te zijn, kan het gebruikt worden om geometrieën vinden waarin turbulentie sterk verminderd is, wat uiteindelijk fusie-energie dichter bij de realiteit kan brengen.


\vspace*{11pt}\noindent



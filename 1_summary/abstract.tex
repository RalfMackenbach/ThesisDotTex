%*********************************************************************************%
\chapter*{Summary} % Try to keep within approx 350 words / one page
\addcontentsline{toc}{chapter}{Summary}
\markboth{Summary}{Summary}


\begin{center}
\rule{\textwidth}{.75pt}\vspace*{1mm}
\textbf{{\Large Available Energy\\[2mm] \textit{A compass for navigating the nonlinear landscape of fusion plasma turbulence}}}
\rule{\textwidth}{.75pt}
\end{center}
\vspace*{2ex}
Nuclear fusion, the process in which light atomic nuclei combine to form heavier nuclei, releases large amounts of energy, with its specific energy (the energy per unit mass of fuel) being roughly a factor $\sim 10^7$ larger than that of gasoline. Furthermore, this process (which powers many of the stars in our universe) does not emit greenhouse gases, and both these qualities make it an attractive energy source. Magnetic confinement fusion attempts to realise this goal by confining the hot fuel, an ionised plasma at roughly 150 million degrees centigrade, with a strong magnetic field of several Tesla. Importantly, this fuel needs to be well insulated so that one can maintain the high temperature required. However, the plasma exhibits turbulence, which degrades the insulation and makes it more difficult to achieve the desired properties. \par
In this thesis, we investigate the merit of a thermodynamic quantity, called available energy (\AE{}), as a measure of turbulence in fusion plasmas. Importantly, a deeper understanding of such turbulence can aid in attaining much-improved fusion reactor performance, hopefully bringing magnetic confinement fusion closer to a reality. Available energy derives its name from the fact that it is an upper bound on the amount of thermal energy that may be extracted from the plasma, subject to a number of constraints, i.e. it is the amount of energy which is {\it available} to be liberated. Although this quantity first appeared in the realm of mathematics nearly a century ago, it has found fairly little use as a turbulence measure. In this thesis, we make a first attempt at such an investigation by specialising to the so-called trapped electrons, which are trapped in a region of weaker magnetic field strength. The regions where such particles are trapped are decided by the magnetic field $\boldsymbol{B}$, and the functional form of this field is often called its geometry. \par 
To calculate the \AE{}, we first develop methods and benchmarks for accurately calculating emblematic quantities of such trapped particles, as they will enter the calculation of \AE{}. With these numerical routines in place, the \AE{} of trapped electrons in general geometries is analytically derived in the limit of a flux tube, a slender domain around a magnetic field line that is everywhere parallel to the magnetic field. This expression for \AE{} is then calculated for a number of geometries for which nonlinear turbulence simulations exist and compared against the results of these simulations. Comparing the turbulent energy flux (i.e., the amount of energy per unit time per unit area that passes through the domain in the fully turbulent state) with \AE{}, we find that the two are correlated via a power law, highlighting that \AE{} points to fundamental properties of the turbulence. To further investigate the various dependencies \AE{} exhibits, we go on to specialise in tokamaks. This allows one to readily investigate how \AE{} changes as one varies the magnetic shear, the pressure gradient, the triangularity, and other figures of merit central in tokamaks. This reproduces some known trends, such as the highly beneficial nature of negative magnetic shear, and allows us to make qualitative statements on the benefits of negative triangularity. \par 
All in all, we find that \AE{} is a useful measure of turbulence, opening up computationally cheap methods to estimate turbulence levels without the need for non-linear simulations. It also provides insight into why some configurations are favourable over others, due to the fact that \AE{} is easy to interpret. Looking further into the future, possible generalisations could extend the domain of utility of \AE{} beyond trapped electrons, making it a useful tool in more general scenarios. If such a generalisation proves successful, finding geometries in which turbulence is much reduced may then readily be achieved, ultimately bringing fusion power closer to reality.

\vspace*{11pt}\noindent
\textbf{Keywords:} \ \ Plasma turbulence, Fusion plasma, Available energy, Plasma thermodynamics, Magnetic confinement fusion, Gyrokinetics
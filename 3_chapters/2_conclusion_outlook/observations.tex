\chapter{Some observations concerning passing electrons}
\vspace*{3mm}
Many generalisations of the \AE{} calculation are possible, and all depend {\it critically} on the chosen invariants, and different choices of invariants have been considered before. The following chapter considers a new, especially natural, generalisation of the results in the current thesis, by investigating passing electrons, and I am grateful for both P. Helander and E. Rodr\'iguez for providing crucial insights here. The generalisation is included, as the calculation of \AE{} is entirely equivalent with only minor interpretative differences.

\section{Invariants on other species}
We have seen in past publications that the \AE{} of trapped electrons serves as a useful measure of gyrokinetic turbulence. Although the research is fruitful, it is lacking in some parts: ions are not accounted for, nor are passing electrons. Thus, it is natural to wonder if \AE{} can be extended to account for these species. One such investigation is presented in \citet{helander2020available}, where the following assumptions are made regarding the various species:
\begin{itemize}
    \item All species obey Liouville's theorem.
    \item Trapped electrons have invariance of $\mu$ and $\mathcal{J}$.
    \item The ions have invariance of $\mu$ alone.
    \item Passing electrons have invariance of $\mu$ and $\psi$.
    \item The electron and ion densities are quasineutral, i.e. the ion charge density exactly balances the electron charge density.
\end{itemize}
The invariance of $\psi$ of passing electrons is motivated by the fast flow of the electrons, which ensures that the radial drift vanishes to the leading order on irrational surfaces \cite{helander2014theory}. However, this constraint is especially restrictive as invariance of $\psi$ makes it impossible to extract energy by flattening the gradient and \AE{} vanishes exactly. However, we know that there may be situations where passing electrons can contribute to various instabilities \cite{jenko2000electron,landreman2015universal,helander2015universal,hardman2022extended,costello2022universal}. Let us somewhat weaken the assumptions imposed on these passing electrons, and there is an especially natural choice one can make, which naturally extends the research presented in this thesis. 

\subsection*{An equivalent $\mathcal{J}$ for passing particles}
Let us return to Eq. \eqref{eq: euler-lagrange for psi}, the Euler-Lagrange equation for the radial coordinate. Restating the result for convenience, we had found the following relationship
\begin{equation}
    \frac{\mathrm{d}}{\mathrm{d}t}\left( \psi + \frac{m v_\parallel}{q} \boldsymbol{b} \cdot \partial_\alpha \boldsymbol{R} \right) = - \frac{\mu}{q} \partial_\alpha B.
\end{equation}
Integrating between two bounce points (where $v_\parallel$ vanishes), we were able to relate the total radial excursion that a trapped particle makes to derivatives of $\mathcal{J}$. Let us now instead consider a passing particle, in a periodic domain (as may be found in a tokamak, or on a rational surface of a stellarator), and we integrate across this domain. Due to this periodicity, we see that
\begin{equation}
    \int_{\rm period} \frac{\mathrm{d}}{\mathrm{d}t} \left( \frac{m \dot{\ell}}{q} \boldsymbol{b} \cdot \partial_\alpha \boldsymbol{R} \right) \mathrm{d} t = 0,
\end{equation}
and, using $E = \mu B + m v_\parallel^2/2$, the Euler-Lagrange equation becomes
\begin{equation}
    q\Delta \psi = \partial_\alpha \int_{\rm period} m v_\parallel \mathrm{d} \ell \equiv \partial_\alpha \mathcal{J}_{p}.
\end{equation}
Therefore, we have found that a passing equivalent of $\mathcal{J}$ behaves entirely analogously to the trapped one. For the binormal coordinate, one finds
\begin{equation}
    q\Delta \alpha = - \partial_\psi \mathcal{J}_p,
\end{equation}
and one again finds that the differential $\mathrm{d}\mathcal{J}_p = 0$, showcasing its invariance. \par 
It is thus a natural choice to impose that $\mu$ and $\mathcal{J}_p$ are conserved for these passing particles, and the calculation of \AE{} is completely equivalent to the one presented for trapped particles. The only interpretative difference is that $\omega_\alpha$ and $\omega_\psi$ now refer to the radial and binormal drift \textit{passing} particles experience in one transit of the periodic domain, where the transit time is given by $\partial_E \mathcal{J}_p$. This equivalence similarly implies that the stabilising properties of maximum-$\mathcal{J}_p$ transfer to the passing population, too. Since $\partial_\alpha \mathcal{J}_p \propto \omega_\psi$ is often much reduced (if not zero) for these passing particles, let us focus on the behaviour of $\partial_\psi \mathcal{J}_p$ of these passing particles.

\section{The maximum-$\mathcal{J}_p$ property}
The analysis of the binormal drift of passing particles is made simpler due to the fact that the singular behaviour bounce-averaging integrals exhibit is not present for passing particles. As such, we may simply investigate the expression for the passing adiabatic invariant and take derivatives without further complications. The passing invariant may be written as
\begin{equation}
    \mathcal{J}_p = \sqrt{2 m E }  \int_{\rm period} \sqrt{1 - \lambda \hat{B}} \: \mathrm{d} \ell,
\end{equation}
with $\lambda = \mu B_0/ E$, $B_0$ is some reference magnetic field, and $\hat{B} = B/B_0$. Taking the derivative with respect to $\psi$, one finds
\begin{equation}
    \partial_\psi \mathcal{J}_p = - \sqrt{\frac{m E}{2}} \int_{\rm period} \frac{\lambda \partial_\psi \hat{B}}{\sqrt{1 - \lambda \hat{B}}} \mathrm{d} \ell.
\end{equation}
The derivative with respect to $E$ may similarly be found as
\begin{equation}
    \left( \partial_E \mathcal{J}_p \right)_{\psi,\alpha,\mu} = \sqrt{\frac{m}{2E}} \int_{\rm period} \frac{\mathrm{d} \ell}{\sqrt{1 - \lambda \hat{B}}},
\end{equation}
and the binormal drift hence reduces to
\begin{equation}
    \omega_\alpha = \frac{E}{q} \frac{\int_{\rm period} (1 - \lambda \hat{B})^{-1/2} \lambda \partial_\psi \hat{B} \: \mathrm{d} \ell }{\int_{\rm period} (1 - \lambda \hat{B})^{-1/2} \: \mathrm{d} \ell}.
\end{equation}
We are now in a position to evaluate the drift. Realise that passing particles have $\lambda \in [0,\hat{B}_{\rm  max}^{-1})$, and it is natural to consider the two limiting cases. For well-passing particles (i.e. $\lambda \rightarrow 0$), we find the leading-order expression
\begin{equation}
    \omega_\alpha (\lambda \rightarrow 0) \approx \frac{\lambda E}{q} \frac{\int_{\rm period} \hat{B} \partial_\psi \hat{B} \: \frac{\mathrm{d} \ell}{\hat{B}} }{\int_{\rm period} \hat{B} \: \frac{\mathrm{d} \ell}{\hat{B}}} \approx \frac{\lambda E}{2q} \frac{\langle \partial_\psi \hat{B}^2 \rangle_{\rm fs}}{\langle\hat{B} \rangle_{\rm fs}},
\end{equation}
where we have introduced the flux-surface averaging operator as $\langle \dots \rangle_{\rm fs}$. This expression is similar to the \textit{magnetic well} term, useful for determining magnetohydrodynamic stability \cite{greene1997brief,landreman2020magnetic,rodriguez2023magnetohydrodynamic}. For favorable magnetodhydrodynamic stability properties, one requires that $D_{\rm well} \propto \langle \partial_\psi \ln \hat{B} \rangle_{\rm fs} > 0$ \cite{helander2014theory}. The maximum-$\mathcal{J}_p$ property requires that $q \omega_\alpha >0 \implies \langle \partial_\psi \hat{B}^2 \rangle_{\rm fs} > 0$, and we thus see a correspondence between the magnetic well and the maximum-$\mathcal{J}$ property. More precisely, if one expands $\hat{B}$ around the smallness in its variation, the criteria to leading order are equivalent, and thus there is a synergy between this magnetohydrodynamic stability criterion and the maximum-$\mathcal{J}_p$ property for passing particles. \par 
Let us next investigate a second limiting case, namely the barely passing particles that have $\lambda \rightarrow \hat{B}_{\rm  max}^{-1}$. Such passing particles spend nearly all their orbit time near the global maximum of $B$ on the flux surface, and as such sample $\partial_\psi B$ only at this location. Denoting $\hat{B}(\ell_{\rm max}) = \hat{B}_{\rm max}$, we then have
\begin{equation}
    q \omega_\alpha (\lambda \rightarrow \hat{B}_{\rm  max}^{-1}) = \frac{E}{\hat{B}_{\rm max}} \partial_\psi \hat{B}(\ell_{\rm max}).
\end{equation}
To investigate the sign of $\partial_\psi B(\ell_{\rm max})$, let us consider the following. 
\begin{enumerate}
    \item Assume that $\partial_\psi B(\ell_{\rm max}) < 0 $ for all $\psi$. This implies that there is a point-maximum in $B$, somewhere on the magnetic axis. However, such a point maximum is impossible as a consequence of Earnshaw's theorem \cite{earnshaw1848nature}, and near the magnetic axis one {\it must} have $\partial_\psi B(\ell_{\rm max}) > 0$.
    \item Let us next assume that $B_{\rm max}$ is unique on the flux surface (this excludes, for example, exactly omnigeneous equilibria). If we then assume that $\partial_\psi B(\ell_{\rm max})$ changes sign for some $\psi > 0$, we have again introduced a local point maximum at the point $\partial_\psi B = 0$. This is once again impossible due to Earnshaw's theorem, and as such we find that $\partial_\psi B > 0 $. If we instead assume that $B_{\rm max}$ is not unique but there is a line of constant $B_{\rm max}$, one simply has one ignorable coordinate and Earnshaw's theorem still holds. So, in this case too, barely passing particles are maximum-$\mathcal{J}_p$.\footnote{I wish to stress that this crucial insight was provided to me by Helander and Rodriguez for which I am thankful, and this finding proved crucial in their investigation of the maximum-$\mathcal{J}$ property in quasi-isodynamic systems which will be published soon.}
\end{enumerate}
We have thus found that the barely passing particles are in all scenarios maximum-$\mathcal{J}_p$. \par 
All in all we have investigated two limiting cases. Well-passing particles are maximum-$\mathcal{J}_p$ if the flux-surface average of $\partial_\psi B^2$ is greater than zero (similar to the magnetic well), and barely passing particles are maximum-$\mathcal{J}_p$ due to the impossibility of point-maxima in $B$. Let us then postulate that, if both limiting cases are maximum-$\mathcal{J}_p$, they are connected monotonically in $\lambda$, and thus all particles are maximum-$\mathcal{J}_p$.

\subsection*{An example case: the limit of the large-aspect ratio tokamak}
Let us, as an example case that showcases our findings from the previous section, calculate the drift in a large-aspect ratio tokamak with circular flux surfaces. In such a limit, we know that the arc length relates the geometric poloidal angle $\theta$ via $\mathrm{d}\ell \approx R_0 \mathrm{d} \theta / \iota$, and the magnetic field is $\hat{B} \approx 1 - \frac{r}{R_0} \cos \theta$ \cite{helander2005collisional}. The passing invariant thus becomes
\begin{equation}
    \mathcal{J}_p \approx \frac{R_0}{\iota} \sqrt{2mE} \int_{-\pi}^{\pi} \sqrt{1 - \lambda (1 - \epsilon\cos \theta)} \: \mathrm{d} \theta,
\end{equation}
where we have introduced $\epsilon = r/R_0$. We may readily take derivatives with respect to $\epsilon$ and $H$ to find the drift, and the result to leading order in the smallness of $\epsilon$ is
\begin{equation}
    q \omega_\alpha = \frac{H}{B_0 R_0 r} \left(\frac{2}{k^2} \left[1 - \frac{E(k)}{K(k)} \right] - 1 \right) \equiv \frac{H}{B_0 R_0 r} \mathcal{F}(k).
\end{equation}
Here we have defined the ``passing'' parameter as $k^2 = 2 \lambda \epsilon / ( 1 + \lambda [ \epsilon - 1 ] )$, which is zero for well-passing particles and one for barely trapped particles. Furthermore, complete elliptic integrals of the first and second kind are defined as $K(k) = \int_{-\pi}^{\pi} (1 - k^2 \sin^2 \theta)^{-1/2} \mathrm{d} \theta$ and $E(k) = \int_{-\pi}^{\pi} (1 - k^2 \sin^2 \theta)^{1/2} \mathrm{d} \theta$. \par 
In terms of the previous section, we find that for around $k^2$ the drift is $\mathcal{F}(k \rightarrow 0) \approx k^2 / 8$. For barely passing particles, one finds $\mathcal{F}(1) = 1$, and intermediate values are included in the plot of $\mathcal{F}(k)$ given in Fig. \ref{fig: F(k)}. For the case given here, it can be seen that all particles are indeed maximum-$\mathcal{J}_p$.
\begin{figure}
    \centering
    \includegraphics[width=0.6\textwidth]{3_chapters/2_conclusion_outlook/Fk.pdf}
    \caption{The passing particle drift as a function of the parameter $k$.}
    \label{fig: F(k)}
\end{figure}

\section{When is conservation of $\mathcal{J}_p$ relevant?}
Finally, let us consider under what conditions invariance of $\mathcal{J}_p$ is relevant for turbulent transport. The particles should reside in the domain long enough so that turbulence may substantially reorder particles, and one may reach a state akin to a ground state. To be somewhat more precise, the typical residence time of a passing particle in a domain where turbulence resides should be comparable to or greater than typical drift-wave turbulence time scales. Evidently, irrational surfaces break such criteria, as particles sample different field lines with different turbulent structures on them after each transit. Hence a minimal condition one should meet is that the flux surface should be a, preferably low-order, rational surface. In order to ensure that neighbouring surfaces also meet such properties, so that the turbulence may flatten gradients across surfaces, the magnetic shear should similarly not be too large. From the discussion and example given in the preceding sections we have postulated that most equilibria satisfy the maximum-$\mathcal{J}_p$ condition, implying stabilising properties of such low-shear rational surfaces. Such surfaces are often found to be regions where internal transport barriers appear, which are regions with much reduced transport and increased gradients \cite{fujita1997internal,turri2008role,ida2018internal}, although it should be noted that $\boldsymbol{E} \times \boldsymbol{B}$-shearing is thought to play an important role here which the current theory does not account for. \par 
All in all, the above gives a straightforward method of generalising the results in the rest of the thesis to account for passing electrons, with some observations of possible use-cases, using the same tools employed in the rest of the thesis. There are, of course, conditions when the presented method falters. Firstly, one should not expect to accurately model electron-temperature-gradient mode driven turbulence with this method, as both the transit time of the passing electrons and the bounce time of trapped electrons become comparable with the typical turbulence timescales. Hence, only in, e.g. trapped electron mode or ion temperature gradient mode-driven scenarios should one expect this measure to be reliable.

%!TEX root = ../../thesis.tex
\chapter[Available energy in tokamaks]{Available energy in tokamaks}
\label{chap: AE-TE}
As we have seen, the \AE{} of trapped electrons showed correlation with gyrokinetic turbulence, and as such it holds some merit as turbulence measure. There are many possible paths which one could explore now (e.g. optimisation, further generalisation, experimental validation), and several are currently being pursued. One especially fruitful research which had been conducted after the found utility of \AE{} concerns tokamaks. \par 
Since the parameter-space of stellarators is especially large, and it may be useful to investigate the dependencies of \AE{} on typical equilibrium parameters, it is useful to specialise the geometry further to tokamaks. More specifically, the parameter space of tokamaks is significantly reduced, one may explore dependencies of the \AE{} on typical parameters such as magnetic shear, pressure gradient, and triangularity. This will allow us to make contact with the large body of literature concerning tokamak research, and one can make both qualitative and quantitative statements concering potential benefits of various equilibrium parameters. This is what is done in the proceeding publication.
\vfill \newpage


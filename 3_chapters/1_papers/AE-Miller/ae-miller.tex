%!TEX root = ../../thesis.tex
\chapter[Available energy in tokamaks]{Available energy in tokamaks}
\label{chap: AE-TE}
As we have seen, the \AE{} of trapped electrons showed correlation with gyrokinetic turbulence and as such serves as a useful measure of turbulence transport. There are many possible paths that one could explore now (e.g. optimisation, further generalisation, experimental validation), and several are currently being pursued. An especially fruitful investigation that had been conducted after the found utility of \AE{}, concerns tokamaks. We are interested in finding dependencies of \AE{} on various parameters of interest, e.g., the pressure gradient, magnetic shear, or shape of the flux surface. In tokamaks, these parameters can be varied independently by employing the Mercier-Luc formalism, where the Grad-Shafranov equation is solved locally to a flux surface \cite{MercierLuc1974}, allowing one to assess the impact of the parameters on \AE{}.\footnote{One can employ a similar method to the Mercier-Luc formalism in stellarators, see e.g. \citet{hegna2000local}.} We employ this method in the following paper, which will allow us to make contact with the large body of literature on tokamak research, and one can make both qualitative and quantitative statements concerning the potential benefits of various equilibrium parameters. This is what we do in the following publication.
\vfill \newpage
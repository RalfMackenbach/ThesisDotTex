\chapter*{Acknowledgements}
\begin{CJK}{UTF8}{min}
{\it``Nou, dit is dan het PhD kantoor! Pak gerust een vrij tafeltje.''} A quick chat and a welcoming {\it bakske koffie} from team fusion on my first day of employment made me feel right at home in the Aquarium; the nickname given to the PhD office, which is surrounded by four glass walls. During my master's degree, I had often seen the {\it fish} hard at work there, doing maths, coding, or (most likely) having a chat about fusion-adjacent topics at the coffee table. Swimming in their wake, eager to start my adventure, I felt both anxious and excited about the twisty-whirly world I was about to enter. And what an adventure it was. Over the past four years I have been lucky enough to meet truly {\it stellar} people, and it is difficult for me to express how much I value their warmth, friendliness, and support (but let me take a swing at it anyway). \par 

This holds true, first and foremost, for my supervisors Josefine Proll and Per Helander. Josefine, only in hindsight do I realise what incredible luck I had to work (and gossip) with you. Throughout the PhD you were always there to guide me through the world of academia, with many words of wisdom and bits of practical advice. In the worst of COVID months, you understood whenever I was an unproductive mess. In the most exciting research months, you would always find interesting new directions to push the research in. Also, which other supervisor would be there to talk about contour integration or the indisputable superiority of the Kyoto Station cheese tarts? There was never a dull moment. I don't think I could have asked for a better supervisor. Per, every visit to Greifswald felt like arriving at my home away from home (with Kaffee und Kuchen). This is due to the warm and welcoming environment you have cultivated in the Stellaratortheorie department. I have learnt a great deal about plasmas, academia, and research in our many chats. Thank you for always having available time and listening to my ideas. I can confidently say that I have learnt what it means to be a good researcher from both of you, a debt I can only hope to repay with many Bach cantatas and matcha lattes. \par 

Then, my alma mater, the friendly fusion folks of Eindhoven. The {\it gezelligheid} in our department is in no small part because of the engine of the fusion group, Clazien Saris and H\'el\`ene Kempermans. Thank you for all the help, taart, and vlaai. Niek and Roger, the main reason I embarked on this adventure to begin with was because of the excellent education you provide at the university and I am grateful to have enjoyed it. Felix, as you may have noticed, I am overjoyed to have another stellarator enthusiast in Eindhoven — thank you for the interesting chats. M.J., you are a true \textsc{gene}-oracle, thanks for all your help. The Fusenet office, Myra and Dario, from lunches to PhD events in exotic places, it was always a blast. Thank you Robert-Jan for our chats, I have learnt a lot about what makes an organisation like ITER tick from you. A similarly {\it gemütlich} atmosphere is present in Greifswald. Vielen Dank Anja, dass Sie mich immer herzlich willkommen geheißen hat. Thanks also to Gareth, Gabe, Sophia, Brennan, Ksenia, Pavel, Beto, Adrian, and Alessandro, our many chats were always a delight. Thank you Thomas, Ich hoffe dass wir noch oft einen Drink am Hafen genießen können. \par 

I have been lucky enough to visit different institutions, and I fondly remember the hospitality of the people there. Professor Nagasaki-sensei and Mayumi-san, thank you for hosting me at Kyoto University and I hope that we will meet again soon. 心から感謝しています. Professor Barnes, thank you for arranging the visit to Oxford; both you and the Oxford group are very welcoming. From amusing epistemic discussions about current political affairs at lunch to the nitty-gritty of plasma physics, I have learnt a great deal in the short time I could work with you, and I hope we may work together more. \par

Fellow PhD students and postdocs, the old movie trope {\it ``the real treasure was the friends we made along the way''} is clich\`e but so true. I will try to keep this as brief as possible, but to use Fermat's words: My gratitude is too large to fit the margins. Alan, though we have only known each other for about 2 years, you feel like a friend I have known for ages and we can talk about anything. Paul M., you have been an invaluable friend; thanks for allowing me to whinge, going on rants on current affairs, or whatever new pop culture things demanded attention. Eindhoven folks — Peter, Arthur, Carlos, Nishith, Bram, Daan, Tijs, Sven, Jos, Maria, Thomas, Torben, Laura, Aaron, Joost, Chris, Fabio, Garud, Sam, Reeve, Timo, Nick, and Maikel; thanks for the borrels, coffees, and chats. Greifswald Leute — Eduardo, Paul C., Linda, Gregor, Bob, Michail, Alistair, Katia, Albert, Yann, Frederik, and Jim-Felix; thanks for taking me kiteboarding, festivalling, and Hafening. Oxford lads and lasses — Robbie, Georgia, Michael, Patrick, Juan, Leonard; thanks for the engaging chats and the nights with pints and dances. \par 

Students whom I have had the pleasure to supervise; Jonathan, Thijs, Ramon, Menno, Marieke, Sjoerd, and Jaime. Although the supervisor-student roles may suggest otherwise, I have learnt as much (if not more) from you as you have from me. Thank you for taking on a project with me and I hope you have found it as pleasant as I have. \par 

Now, this thesis would not have been possible without the support of my friends, and it is impossible to list all those who have contributed in some meaningful way (but let me try). Nirbhav, thanks for being the best friend I could ask for. Although there is the slight inconvenience of an Atlantic ocean separating us, this has never diminished our friendship. Thanks Chris, for flying across the two oceans just to hang. You are a true friend. Let me switch to Dutch now to thank my Dutchies. Saar en Juul, bedankt voor alles. Van Cluedo avondjes tot paper-peper-parties, jullie zijn zo ontzettend aardig. Ik ben blij dat ik jullie tot mijn vrienden mag rekenen. Igor, maat, ik kan nauwelijks uitdrukken hoe chill het is dat je weer in de lage landen bent. Ik hoop dat we vele road-tripjes gaan maken, en techno-knallers in elkaar gaan zetten (wanneer is de Sainte Marie comeback tour?). Robin, Anne, Gilles, Nick, Raynor, Sahil, Chris, Lisalotte, Saara, Lars, Guido, Stan, Alex, de Leeuwen's, alle Eva's, Michael, Simon; jullie zijn toppertjes. Mijn wakeboard homies; Koen jij bent en blijft menne maat. Damian, Demi, Bogus, Stoffel, Aschwin, et. al., de baan is mijn tweede huis dankzij jullie. Tom, op nog vele {\it skypeborrels}. Nu houd ik op, anders blijf ik bezig. \par 

Als laatste wil ik mijn familie bedanken, die mij altijd support. Bedankt Rickert, dat je de beste broer (en bovenbuurman) bent die er is — op nog vele koffietjes bij het koffiehuisje. Bedankt Roelert, je bent de beste papa voor Bella \& Duuk. Super Smash Bros ben je niet zo goed in, maar je kan niet alles hebben natuurlijk. Bedankt Tamara, de leukste mama voor Bella \& Duuk die er is. Thanks Karyn, you truly are a wonderful addition to our family. Bedankt Bella, ik hoop dat we nog vaak gaan boeballe en swemme. Duuk, ik kan niet wachten tot ik mag helpen bij jouw wiskunde huiswerk. En als laatste natuurlijk, bedankt paps en mams. Jullie staan altijd voor me klaar, welke keuze ik ook maak, van Greifswald tot Princeton, van Kioto tot Eindhoven. \newline \newline
Ik houd van jullie.
\end{CJK}
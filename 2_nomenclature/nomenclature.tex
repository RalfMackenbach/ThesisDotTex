\newenvironment{Nomen}
    {\vspace*{-3mm}\begin{center}
    \begin{longtable}{p{.1\textwidth} p{.83\textwidth}}
    }
    {
    \end{longtable}
    \end{center}\vspace*{-1.2cm}
    }
\newcommand{\AddSymbol}[2]{#1 & #2 \\}




%*********************************************************************************%
\chapter*{Nomenclature}
\addcontentsline{toc}{chapter}{Nomenclature}
\markboth{Nomenclature}{Nomenclature}


%*********************************************************************************%
\section*{Acronyms}

\begin{Nomen}
\AddSymbol{\AE{}}{Available Energy}
\AddSymbol{TEM}{Trapped electron mode}
\AddSymbol{ITG}{Ion temperature gradient mode}
\AddSymbol{ETG}{Electron temperature gradient mode}
\AddSymbol{UI}{Universal instability}
\AddSymbol{BAD}{Bounce-averaged drift}
\AddSymbol{QS}{Quasi-symmetric}
\AddSymbol{QI}{Quasi-isodynamic}
\end{Nomen}


%*********************************************************************************%
%\newpage
\section*{Operators and  symbols}

\begin{Nomen}
\AddSymbol{$(\partial_x f)_{\boldsymbol{y}}$}{The partial of $f$ with respect to $x$, at constant $\boldsymbol{y}$}
\AddSymbol{$x \in \mathcal{X}$}{The element $x$ is a member of the set $\boldsymbol{X}$}
\AddSymbol{$\mathcal{Y} \subseteq \mathcal{X}$}{The set $\mathcal{Y}$ is a subset of the set $\mathcal{X}$}
\AddSymbol{$\dot{x}$}{The temporal derivative of $x(t)$, i.e. $\dot{x} \equiv \partial_t x$}
\AddSymbol{$\boldsymbol{B}$}{The magnetic field}
\AddSymbol{$\boldsymbol{A}$}{The magnetic potential which satisfies $\nabla \times \boldsymbol{A} = \boldsymbol{B}$}
\AddSymbol{$B$}{The magnetic field strength, i.e. $B \equiv |\boldsymbol{B}|$}



\end{Nomen}

%\thispagestyle{empty}
